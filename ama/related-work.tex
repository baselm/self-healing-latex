\section{Related work}
\label{sec:related-work}

Self-adaptive software is characterised by a number of properties best referred to as   autonomic \cite{jelasityself}. These the `self-* properties' include Self-organisation, Self-healing, Self-optimisation and Self-protection  \cite{horn:2001p3735}. Self-organising is the capability of a software to self-reconfiguring itself automatically and dynamically in response to changes including installing, updating, integrating, and composing/decomposing software entities \cite{Salehie:2009p3693}. Self-healing is the capability of a software of discovering, diagnosing and reacting to disruptions. It can also anticipate potential problems and, accordingly, take suitable actions to prevent a failure \cite{horn:2001p3735}. 
Self-healing aspects of Microservices architectures requires a decision-making strategy that can work in real-time. This is essential for the software to reason about its own state and its surrounding environment in a closed control loop model and act accordingly \cite{Cheng:2008p3708}.  
Typically, a self-adapting system follows MAPE-K model (Monitor-Analyse-Plan-Execute over a shared Knowledge) and implements the following main functions: 
\begin{enumerate}
\item  \textbf{Monitor:} Gathering of data related to the surrounding context (Context Sensing);
\item \textbf{Analyze:} Context observation and detection;
\item \textbf{Plan:} Dynamic decision making;
\item  \textbf{Execute:} Adaptation execution to achieve the adaptation objectives defined as Quality of Servcie (QoS);
\item Verification and validation of the applied adaptation actions in terms of its ability to meet the adaptation objectives and achieve the desired QoS and at the same time maintaining the architecture state.
\end{enumerate}

This section focuses on discussing the related work of anomaly detection and dynamic decision making based on multi-dimensional utility-based model. However, there is many approaches are used for achieving high level of self-adaptability though Context sensing involving context collection, observation and detection of contextual changes in the operational environment. Also, the ability of the system to dynamically adjust its behaviour can be achieved using parameter-tuning \cite{Cheng:2009p3902}, component-based composition \cite{MariusMikalsen:2005ur}, or Middleware-based  approaches \cite{CheungFooWo:2007p1692}. Another important aspect of self-adaptive system is related to its ability to validate and verify the adaptation action at runtime based on Game theory \cite{Wei:2016ge}, Utility theory as in \cite{Menasce:2007vq,KonstantinosKakousis:2008ub}, or Model driven approach as in \cite{Sama:2008p3765}.

Context information (1) refers to any information that is computationally accessible and upon which behavioural variations depend \cite{Hirschfeld:2008p1620}. Context observation and detection approaches (2) are used to detect abnormal behaviour within the microservices architecture at run-time. Related work in context modelling \cite{Strang:2004p3770}, context detection and engineering self-adaptive software system are discussed in \cite{Salehie:2009p3693,Cheng:2008p3708,RogeriodeLemos:2011tj}.  
In dynamic decision making and context reasoning (3), the architecture should be able to monitor and detect normal/abnormal behaviour by continuously monitoring the contextual information found in the microservices cluster. There are two phases for detecting anomalies in a software system: a training phase which involves profiling the normal behaviour of the system; a second phase aimed at testing the learned profile of the system with new data and employing it to detect normal.abnormal behaviours \cite{Patcha:2007hja}. Three major techniques for anomaly detection have emerged from the literature: statistical anomaly detection, data-mining and machine-learning based techniques \cite{Patcha:2007hja}. 
Within the statistical methods, a system observes the activity of the system and generates profiles of  system metrics to represent its behaviour. The system profile includes performance measures of the system resources such as CPU and Memory. For each measure, a separate profile is stored. Then, the current readings of the system are profiled and compared against the memorised past profile to calculate an anomaly score. This score is calculated by comparing all measures within the profile against a threshold specified by the developer. Once the system detects that the current readings of the system  are higher than this threshold, then these will be automatically categorised as  intrusions thus triggering an alert \cite{manikopoulos2002network,kruegel2003anomaly}. 
Various statistical anomaly detection systems have been proposed and they have some advantages \cite{lunt1992real,denning1985requirements,anderson1995next,roesch1999snort,maxion1990case}. 
One of this is that they can detect an anomaly without prior knowledge of the system. This can mitigate the common problem of a cold start found in machine learning techniques. Additionally, statistical anomaly detection provides accurate notifications of malicious attacks that occurred over long periods of times and it performs better in detecting denial-of-service attacks \cite{Patcha:2007hja}. 
However, a disadvantage is that a skilled attacker might train a statistical anomaly detection system to accept the abnormal behaviour as normal. It is difficult to determine the thresholds that make a balance between the likelihood of a false negative (the system fails to identify an activity as an abnormal behaviour) and the likelihood of a false positive (false alarms). Statistical methods need an accurate  model with a precise distribution of all measures. In practice, the behaviour of virtual machines/computers cannot be entirely be modelled using solely statistical methods.

Data mining is about finding insights which are statistically reliable, unknown previously, and actionable from data \cite{phua2010comprehensive}. The dataset must be available, relevant, adequate, and clean. The data mining process involves discovering a novel, distinguished and useful data pattern in large datasets to extract hidden relationships and information about the data. In general, there are two issues involved in the use of data mining in an intrusion detection system. First, there is a lack of a large dataset to be used by the algorithm containing lots of information about the Microservices architecture. Second, few approaches were targeting the Intrusion Detection System in Microservices architecture \cite{phua2010comprehensive}

Data mining based intrusion detection systems have three major difficulties which prevent them from being widely adopted in Microservices architecture \cite{lunt1992real,Patcha:2007hja}. Firstly, the low accuracy of detecting an anomaly \cite{gupta2016network,Patcha:2007hja}, as the data mining process would require large dataset with longer time interval to be able to improve the accuracy of detection 
Most data mining techniques are very heavy on the computational resources, so makes the efficiency has high impact on the computational resources through the training, monitoring and detection phase.As most data mining techniques heavily rely on computational resources, this negatively influeence their adoption  in a Microservices architecture  \cite{Patcha:2007hja}. Additionally, usually a data mining method used to classify an attack within a specific system cannot be successfully employed within another system for the same purpose. This because the process of training, testing the model and performing classification of anomalies needs to be repeated with different data or architecture \cite{Buczak:2016kt}.  

Machine learning, in the context of  anomaly detection, can allow the creation of  software system able to learn and improve its detection accuracy over time \cite{bujlow2012method}.
Machine learning-based anomaly detection models aims to detect anomalies similar to statistical and data mining approaches. However, unlike them latter which tend to focus on understanding the process that generated the data, the former are data-driven and are mainly focus on training a model based exclusively on past data \cite{Patcha:2007hja}. This means that, when additional and new data is provided they can intrinsically change their detection strategy and classify significant deviations from the normal behaviour of an underlying software program.
An application of Machine Learning which enables the Microservices cluster to distinguish between normal and abnormal behaviour in the data can be found in \cite{Buczak:2016kt}. 
In general, Intrusion Detection Systems (IDS) uses a combination of clustering and classification algorithms to detect anomalies. The clustering algorithm is used to cluster the dataset and label them. Then, a decision tree algorithm can be used to distinguish between normal and abnormal behaviour.
Golmah \cite{golmah2014efficient} suggested the use of an effective classification model to identify normal and abnormal behaviour in network-based IDS. The usage of Machine Mearning algorithm  in this context can be found in  \cite{golmah2014efficient,Amudhavel:2016kj,haq2015application,Buczak:2016kt,doelitzscher2012agent}. Due to the  opening deployment and limited resources found in a microservices cluster, it is very important to use a lightweight approach to data clustering and classification as suggested in \cite{roesch1999snort,li2006lightweight,snapp1991dids}.
Other Machine Learning-based techniques have been employed in \cite{MdFudzee:2008p3737,pajouh2016two,hodo2016threat}. Most of these techniques are targeting network based attacks in distributed environments. However, the techniques described in \cite{MdFudzee:2008p3737,pajouh2016two,hodo2016threat} pay less attention to induce a model of IDS that can optimise the computational resources and has less impact on the deployment mechanisms. 

Due to this issue, this research focuses on proposing an anomaly detection mechanism that is more suitable for the Microservices architecture and can be easily deployed with less footprints on the limited resources found in the tiny container. 

Several machine learning algorithms have been used in clustering and classifying the network events in IDS. For distributed systems and a wireless sensor network a different approach was used, as shown in \cite{Mishra:2009p3734,lee1999data,li2006lightweight}. Al-Yaseen, Othman, and Nazri \cite{al2017multi} proposed a hybrid mechanism built upon a modified K-means and the C4.5 learning algorithm. Solanki and Dhamdhere \cite{solankiintrusion} used K-means jointly with a Support Vector Machine classifier. The main limitation of K-means clustering is the need to perform initial selection of the data before the clustering can be started. 
 
 Numenta Platform for Intelligent Computing (NUPIC) is based on the Hierarchical Temporal Memory (HTM) model proposed in \cite{Hawkins:2007fi}. HTM has been experimentally applied to image recognition \cite{vskoviera2013image}, natural language processing \cite{arel2010deep}, and most importantly in anomaly detection \cite{ziabaryhlmt,DBLP:journals/corr/LavinA15,DBLP:journals/corr/AhmadP16}.
A novel approach for anomaly detection  in real-time streaming data proposed in \cite{DBLP:journals/corr/AhmadP16,DBLP:journals/corr/LavinA15}. The proposed system based on the HTM model claimed to be efficient and tolerant to noisy data. Most importantly it offers continuous monitoring of real-time data and adapts the changes of the data statistics. It also detects very subtle anomalies with a very minimum rate of false positives. In a recent study, Ahmad et al. \cite{AHMAD2017134} proposed an updated version of the anomaly detection algorithm with the introduction of the anomaly likelihood concept. 
The anomaly score calculated by the NUPIC anomaly detection algorithm represents an immediate calculation of the predictability of the current input stream. This approach works very well with predictable scenarios in many practical applications. As there is no noisy and unpredictable data found the raw anomaly score was gives an accurate prediction of false negatives. However, the changes in predictions would lead to revealing anomalies in the system’s behaviour. Instead of using the raw anomaly score, Ahmad et al. \cite{AHMAD2017134} proposed a method for calculating the anomaly likelihood by modelling the distribution of anomaly scores and using the distribution to check the likelihood of the current state of the system to identify anomalous behaviour. The anomaly likelihood refers to a metric which defines how anomalous the current state is based on the prediction history calculated by the HTM model. So, the anomaly likelihood is calculated by maintaining a window of the last raw anomaly scores and then calculating the normal distribution over the last obtained values, then the most recent average of anomalies is calculated using the Gaussian tail probability function (Q-function) \cite{craig1991new}.
 
 

 Each service running in the Microservices cluster could be scaled in/out based on the demand issued by end-users, the orchestration algorithm, or the load balancer running on the cluster manager \cite{stubbs2015distributed}. This means each service performance and behaviour continuously changes overtime, which makes it a challenge to use a statistical model to identify and detect anomalous behaviour. A sudden peak of the CPU some of the time could be considered to be as an anomaly as it might be an action issued by the cluster manager to meet recent high demand. The performance of the cluster nodes could fluctuate around the demand to accommodate scalability, orchestration and load balancing issued by cluster managers. This requires a model that is able to detect anomalies in real-time and generate a high rate of accuracy in detecting any anomalies and a low rate of false alarms. In addition, there will be a set of variations that can be used by the system to adapt the changes in its operational environment. This requires a dynamic decision making that can calculate the utility of all possible adaptation actions based on the architecture constraints (i.e. number of replicas, number of nodes, desired objectives, metrics thresholds), anomaly score of the detected conditions (CPU, Memory, DISK I/O, Network I/O), and the confidence and accuracy of the anomaly score of the detected abnormal behaviour, and the desired/predicted cluster state. Then, the adaptation manager will execute the adaptation action and verifies its successfulness over the cluster architecture. Also, the adaptation manager will be able to self-tune and self-adjust the architecture parameters to meet high/low demand for services. Finally, the architecture will preserve the cluster state through the adaptation cycle (monitoring, observing, detecting, , reacting, and verifying).

 This research focuses on finding a method to continuously observe and monitor the swarm cluster and be able to detect anomalous behaviour with a high accuracy and generate a low rate of false alarms. Then provide the architecture with adaptation strategies with high utility to reason about the detected anomalies and be able to self-adjust the architecture parameters and verifies the adaptation actions at runtime without human intervention.

For this aim, This research focuses on proposing a model that can continuously observe and monitor the microservices architecture and be able to detect anomalous behaviour with high accuracy and generate low rate of false alarms. At the same time, the architecture should be able to respond to True positive alarms by suggesting a set of adaptation policies (adaptation strategy), that can be deployed in the cluster to achieve high level of self-healing in response to changes in its operating environment. Because of the uniqueness of streaming data found in microservices cluster, the design of self-healing microservices architecture should meets the following requirements: 
\begin{enumerate}
  \item The system should be able to operate over real-time data (no look-ahead). 
  \item The algorithm must continuously monitor and learn about the behaviour of the cluster. 
  \item The algorithm must be implemented with an automatic unsupervised learning technique, so it can continuously learn new behaviour and anomalies in real-time. 
  \item The algorithm must be able to adapt the changes of the operating environment and provides adaptation strategy that can be orchestrated over the cluster nodes. 
  \item The algorithm should be able to detect anomalies as early as possible before the anomalous behaviour is interrupting the functionality of the running services in the cluster. 
  \item The proposed model should minimises the false positives (False Alarms) rate and the false negatives rate. If the system identifies a normal behaviour as an attack, this attempt should be classified as a False Positives (False Alarm). 
  \item The proposed model should offer a high detection rate, better accuracy and a lower false alarm rate.
  \item The proposed model should offer consistence adaptation strategy, and preserve the cluster state and it should offer the architecture with a roll back strategy in case the adaptation action failed. 
  \item One important aspect of a self-healing Microservices architecture is the ability to continuously monitor the operational environment, detect and observe anomalous behaviour, and provide a reasonable policy for self-scaling, self-healing, and self-tuning the computational resources to adapt a sudden changes in its operational environment dynamically at rune-time.  
\end{enumerate}