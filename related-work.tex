\section{Related work}
\label{sec:related-work}
\subsection{Self-adaptive Software}
Self-adaptive software is characterised by a number of properties best referred to as autonomic \cite{jelasityself}. These the `self-* properties' include Self-organisation, Self-healing, Self-optimisation and Self-protection  \cite{horn:2001p3735}. Self-healing architecture refers to  the capability of discovering, diagnosing and reacting to disruptions. It can also anticipate potential problems and, accordingly, take suitable actions to prevent a failure \cite{horn:2001p3735}. Self-healing aspects of Microservices architectures requires a decision-making strategy that can work in real-time. This is essential for Microservice to reason about its own state and its surrounding environment in a closed control loop model and act accordingly \cite{Cheng:2008p3708}.  
Typically, a self-adapting system follows MAPE-K model (Monitor-Analyse-Plan-Execute over a shared Knowledge). a self-healing system should implements MAPE-K model including: Gathering of data related to the surrounding context (Context Sensing); Context observation and detection; Dynamic decision making; Adaptation execution to achieve the adaptation objectives defined as QoS; Verification and validation of the applied adaptation actions in terms of its ability to meet the adaptation objectives and meet the desired QoS.
\subsection{Context Sensing}

 However, there is many approaches are used for achieving high level of self-adaptability though Context sensing involving context collection, observation and detection of contextual changes in the operational environment \cite{Strang:2004p3770}. Also, the ability of the system to dynamically adjust its behaviour can be achieved using parameter-tuning \cite{Cheng:2009p3902}, component-based composition \cite{MariusMikalsen:2005ur}, or Middleware-based  approaches \cite{CheungFooWo:2007p1692}. Another important aspect of self-adaptive system is related to its ability to validate and verify the adaptation action at runtime based on Game theory \cite{Wei:2016ge}, Utility theory as in \cite{Menasce:2007vq,KonstantinosKakousis:2008ub}, or Model driven approach as in \cite{Sama:2008p3765}.

Context information (1) refers to any information that is computationally accessible and upon which behavioural variations depend \cite{Hirschfeld:2008p1620}. Context observation and detection approaches (2) are used to detect abnormal behaviour within the microservices architecture at run-time. Related work in context modelling, context detection and engineering self-adaptive software system are discussed in \cite{Salehie:2009p3693,Cheng:2008p3708,RogeriodeLemos:2011tj,Strang:2004p3770}.  
In dynamic decision making and context reasoning (3), the architecture should be able to monitor and detect normal/abnormal behaviour by continuously monitoring the contextual information found in the Microservices cluster. 

\subsection{Adaptation Planing and Execution}
In Microservices cluster, the performance of the cluster nodes could fluctuate around the demand to accommodate scalability, orchestration and load balancing issued by cluster leader. This requires a model that is able to detect anomalies in real-time and generate a high rate of accuracy in detecting any anomalies and a low rate of false alarms. In addition, there will be a set of variations that can be used by the system to adapt the changes in its operational environment. This requires a dynamic decision making that can calculate the utility of all possible adaptation actions based on the architecture constraints (i.e. number of replicas, number of nodes, desired objectives, metrics thresholds), anomaly score of the detected conditions (CPU, Memory, DISK I/O, Network I/O), and the confidence and accuracy of the anomaly score of the detected abnormal behaviour, and the desired/predicted cluster state. Then, the adaptation manager will execute the adaptation action and verifies its successfulness over the cluster architecture. Also, the adaptation manager will be able to self-tune and self-adjust the architecture parameters to meet high/low demand for services. Finally, the architecture will preserve the cluster state through the adaptation cycle (monitoring, observing, detecting, reacting, and verifying). This research focuses on finding a method to continuously observe and monitor the swarm cluster and be able to detect anomalous behaviour with a high accuracy and generate a low rate of false alarms. Then provide the architecture with adaptation strategies with high utility to reason about the detected anomalies and be able to self-adjust the architecture parameters and verifies the adaptation actions at runtime without human intervention.


\subsection{Anomaly Detection}

There are two phases for detecting anomalies in a software system: a training phase which involves profiling the normal behaviour of the system; a second phase aimed at testing the learned profile of the system with new data and employing it to detect normal/abnormal behaviours \cite{Patcha:2007hja}. 

Three major techniques for anomaly detection have emerged from the literature: statistical anomaly detection, data-mining and machine-learning based techniques. 

Within the statistical methods, a system observes the activity of the system and generates profiles of  system metrics to represent its behaviour. The system profile includes performance measures of the system resources such as CPU and Memory. For each measure, a separate profile is stored. Then, the current readings of the system are profiled and compared against the memorised past profile to calculate an anomaly score. This score is calculated by comparing all measures within the profile against a threshold specified by the developer. Once the system detects that the current readings of the system  are higher than this threshold, then these will be automatically categorised as  intrusions thus triggering an alert \cite{kruegel2003anomaly}. 

Various statistical anomaly detection systems have been proposed and they have some advantages \cite{anderson1995next,roesch1999snort}. 
One of this is that they can detect an anomaly without prior knowledge of the system. This can mitigate the common problem of a cold start found in machine learning techniques. Additionally, statistical anomaly detection provides accurate notifications of malicious attacks that occurred over long periods of times and it performs better in detecting denial-of-service attacks \cite{Patcha:2007hja}. 
However, a disadvantage is that a skilled attacker might train a statistical anomaly detection system to accept the abnormal behaviour as normal. It is difficult to determine the thresholds that make a balance between the likelihood of a false negative (the system fails to identify an activity as an abnormal behaviour) and the likelihood of a false positive (false alarms). Statistical methods need an accurate  model with a precise distribution of all measures. In practice, the behaviour of virtual machines/computers cannot be entirely be modelled using solely statistical methods.

With regard to data-mining approaches, data-mining is about finding insights which are statistically reliable, unknown previously, and actionable from data \cite{phua2010comprehensive}. The dataset must be available, relevant, adequate, and clean. The data mining process involves discovering a novel, distinguished and useful data pattern in large datasets to extract hidden relationships and information about the data. In general, there are two issues involved in the use of data mining in an anomaly detection system. First, there is a lack of a large dataset to be used by the algorithm containing lots of information about the architecture. Second, few approaches were targeting the anomaly detection system in Microservices architecture \cite{phua2010comprehensive}. Data mining based anomaly detection systems have three major difficulties which prevent them from being widely adopted in Microservices architecture \cite{Patcha:2007hja}. Firstly, the low accuracy of detecting anomalous behaviour \cite{gupta2016network,Patcha:2007hja}, as the data mining process would require large dataset with longer time interval to be able to improve the accuracy of detection. Most data mining techniques are heavily on computational resources, this negatively influences their adoption in a Microservices architecture \cite{Patcha:2007hja}. Additionally, usually a data mining method used to classify an attack within a specific system cannot be successfully employed within another system for the same purpose. This because the process of training, testing the model and performing classification of anomalies needs to be repeated with different data or architecture \cite{Buczak:2016kt}.  

Machine learning, in the context of  anomaly detection, can allow the creation of  software system able to learn and improve its detection accuracy over time \cite{bujlow2012method}.
Machine learning-based anomaly detection models aims to detect anomalies similar to statistical and data mining approaches. However, unlike them latter which tend to focus on understanding the process that generated the data, the former are data-driven and are mainly focus on training a model based exclusively on past data \cite{Patcha:2007hja}. This means that, when additional and new data is provided they can intrinsically change their detection strategy and classify significant deviations from the normal behaviour of an underlying software program.
An application of Machine Learning which enables the Microservices cluster to distinguish between normal and abnormal behaviour in the data can be found in \cite{Buczak:2016kt}. 
In general, anomaly detection systems uses a combination of clustering and classification algorithms to detect anomalies. The clustering algorithm is used to cluster the dataset and label them. Then, a decision tree algorithm can be used to distinguish between normal and abnormal behaviour.
Golmah \cite{golmah2014efficient} suggested the use of an effective classification model to identify normal and abnormal behaviour in network-based anomaly detection. The usage of Machine Mearning algorithm  in this context can be found in  \cite{golmah2014efficient,haq2015application,Buczak:2016kt}. Due to the  opening deployment and limited resources found in a Microservices cluster, it is very important to use a lightweight approach of data clustering and classification. Due to this issue, this research focuses on proposing an anomaly detection mechanism that is more suitable for the Microservices architecture and can be easily deployed with less footprints on the limited resources found in the tiny containers running in Microservices cluster . 

Numenta Platform for Intelligent Computing (NUPIC) is based on the Hierarchical Temporal Memory (HTM) model proposed in \cite{Hawkins:2007fi}. HTM has been experimentally applied in real-time anomaly detection of streaming data in \cite{DBLP:journals/corr/AhmadP16,DBLP:journals/corr/LavinA15}. The proposed system based on the HTM model claimed to be efficient and tolerant to noisy data. Most importantly it offers continuous monitoring of real-time data and adapts the changes of the data statistics. It also detects very subtle anomalies with a very minimum rate of false positives. In a recent study, Ahmad et al. \cite{AHMAD2017134} proposed an updated version of the anomaly detection algorithm with the introduction of the anomaly likelihood concept. 
The anomaly score calculated by the NUPIC anomaly detection algorithm represents an immediate calculation of the predictability of the current input stream. This approach works very well with predictable scenarios in many practical applications. As there is no noisy and unpredictable data found, the raw anomaly score gives an accurate prediction of false negatives. However, the changes in predictions would lead to revealing anomalies in the system’s behaviour. Instead of using the raw anomaly score, Ahmad et al. \cite{AHMAD2017134} proposed a method for calculating the anomaly likelihood by modelling the distribution of anomaly scores and using the distribution to check the likelihood of the current state of the system to identify anomalous behaviour. The anomaly likelihood refers to a metric which defines how anomalous the current state is based on the prediction history calculated by the HTM model. So, the anomaly likelihood is calculated by maintaining a window of the last raw anomaly scores and then calculating the normal distribution over the last obtained/trained values, then the most recent average of anomalies is calculated using the Gaussian tail probability function (Q-function) \cite{craig1991new}.

\subsection{Reinforcment Learning}
\textit{The goal of reinforcement learning (RL) is to learn a policy that decides sequential actions by maximizing the cumulative future rewards [30]. Recent trends [23, 32, 28, 27] in RL field is to combine the deep neural networks with RL algorithms by representing RL models such as value function or policy. By resorting of the deep features, many difficult problems such as playing Atari games [23] or Go [27] can be successfully solved in semi-supervised setting. Also, several methods were proposed to solve the computer vision problems, such as object localization [3] or action recognition [16], by employing the deep RL algorithms.
There are two popular approaches in deep RL algorithms: Deep Q Networks (DQN) and policy gradient. DQN is a form of Q-learning with function approximation us- ing deep neural networks. The goal of DQN is to learn a state-action value function (Q), which is given by the deep networks, by minimizing temporal-difference errors [23]. Based on the DQN algorithm, various network architectures such as Double DQN [32] and DDQN [37] were proposed to improve performance and keep stability.
Policy gradient methods directly learn the policy by optimizing the deep policy networks with respect to the expected future reward using gradient descent. Williams et al. [38] proposed REINFORCE algorithm simply using the immediate reward to estimate the value of the policy. Silver et al. [28] proposed a deterministic algorithm to improve the performance and effectiveness of the policy gradient in high-dimensional action space. In the work of Silver et al. [27], it is shown that pre-training the policy networks with supervised learning before employing policy gradient can improve the performance. In tracking problem, we train the proposed network with supervised learning to learn the appearance characteristics of the target objects, and train action dynamics of the tracking target with reinforcement learning using policy gradient method.}
